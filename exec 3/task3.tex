\documentclass[article,11pt]{article}
\usepackage{fullpage}
\usepackage{url}
\usepackage[english]{babel}
\usepackage[utf8]{inputenc}
\usepackage{amsmath}
\usepackage{graphicx}
\usepackage{bm, amssymb}
\usepackage{subfigure}
\renewcommand*{\thesection}{Problem~\arabic{section}:}
\renewcommand{\labelenumi}{(\alph{enumi})}

\newcommand{\Tt}[0]{\boldsymbol{\theta}}
\newcommand{\XX}[0]{{\cal X}}
\newcommand{\ZZ}[0]{{\cal Z}}
\newcommand{\vx}[0]{{\bf x}}
\newcommand{\vv}[0]{{\bf v}}
\newcommand{\vu}[0]{{\bf u}}
\newcommand{\vs}[0]{{\bf s}}
\newcommand{\vm}[0]{{\bf m}}
\newcommand{\mX}[0]{{\bf X}}
\newcommand{\vw}[0]{{\bf w}}
\newcommand{\D}[0]{{\mathcal{D}}}
\newcommand{\mP}{\mathbf{P}}
\newcommand{\E}[0]{{\mathbb{E}}}
\newcommand{\vy}[0]{{\bf y}}
\newcommand{\vb}[0]{{\bf b}}
\newcommand{\vr}[0]{{\bf r}}
\newcommand{\vz}[0]{{\bf z}}
\newcommand{\N}[0]{{\mathcal{N}}}
\newcommand{\vc}[0]{{\bf c}}
\newcommand{\norm}[1]{\Vert  {#1} \Vert}
%\newcommand{\XX}[0]{X}
%\newcommand{\ZZ}[0]{Z}
\newcommand{\bmx}[0]{\begin{bmatrix}}
\newcommand{\emx}[0]{\end{bmatrix}}
%\newcommand{\norm}[1]{\lVert#1\rVert}
\newcommand{\T}[0]{\text{T}}
\DeclareMathOperator*{\rank}{rank}
\newcommand\defeq{:=}
\newcommand\eigvecS{\hat{\mathbf{u}}} 
\newcommand\eigvecCov{\mathbf{u}} 
\newcommand{\featuredim}{d}
\newcommand{\samplesize}{N}
\newcommand{\sampleidx}{i} 
\newcommand{\target}{y}
\newcommand{\error}{E}
\newcommand\dataset{\mathbb{X}}
\newcommand{\inspace}{\mathcal{X}}
\newcommand{\sigmoid}{\sigma}
\newcommand{\outspace}{\mathcal{Y}}
\newcommand{\hypospace}{\mathcal{H}}
\newcommand{\emperror}{\mathcal{E}}
\newcommand{\generror}{\mathcal{E}}
\DeclareMathOperator{\supp}{supp}
%\newcommand{\loss}[3]{L({#1},{#2},{#3})} 
\newcommand{\loss}[2]{L({#1},{#2})} 
\newcommand{\determinant}[1]{{\rm det}({#1})} 
\newcommand{\prob}[1]{P({#1})} 
\newcommand{\pdf}[1]{p({#1})} 
\def \expect {\mathsf{E} }
\DeclareMathOperator*{\argmax}{argmax}
\DeclareMathOperator*{\argmin}{argmin}
%\def \prob {{\rm P} }




\title{CS-E3210- Machine Learning Basic Principles \\ Home Assignment 3 - ``Classification''}
\begin{document}
\date{}
\maketitle

Your solutions to the following problems should be submitted as one single pdf which does not contain 
any personal information (student ID or name).  The only rule for the layout of your submission is that for each 
problem there has to be exactly one separate page containing the answer to the problem. You are welcome to use the \LaTeX-file underlying this pdf, 
available under \url{https://version.aalto.fi/gitlab/junga1/MLBP2017Public}, and fill in your solutions there. 

\newpage

\section{Logistic Regression - I}
Consider a binary classification problem where the goal is classify or label a webcam snapshot into ``winter'' ($y=-1$) or ``summer'' ($y=1$) based on the feature vector 
$\vx=(x_{\rm g},1)^{T} \in \mathbb{R}^{2}$ with the image greenness $x_{\rm g}$. A particular classification 
method is logistic regression, where we classify a datapoint as $\hat{y}=1$ if $h^{(\vw)}(\vx)=\sigma(\vw^T\vx) > 1/2$ and $\hat{y}=-1$ otherwise. Here, we used the sigmoid function 
$\sigma(z) = 1/(1+\exp(-z))$. 

The predictor value $h^{(\vw)}(\vx)$ is interpreted as the probability of $y=1$ given the knowledge of the feature vector $\vx$, i.e., $P(y=1|\vx;\vw) = h^{(\vw)}(\vx)$. Note that 
the conditional probability $P(y=1|\vx;\vw)$ is parametrized by the weight vector $\mathbf{w}$.
We have only $N=2$ labeled data points with features $\vx^{(1)}, \vx^{(2)}$ and labels $y^{(1)}=1, y^{(2)}=-1$ at our disposal in order to find a good choice for $\vw$. 
Let $\vw_{\rm ML}$ be a vector which satisfies 
\vspace*{-3mm}
\begin{equation} 
P(y\!=\!1|\vx^{(1)};\vw_{\rm ML}) P(y\!=\!-1|\vx^{(2)};\vw_{\rm ML}) = \max_{\vw \in \mathbb{R}^{2}} P(y\!=\!1|\vx^{(1)};\vw) P(y\!=\!-1|\vx^{(2)};\vw). \nonumber
\vspace*{-3mm}
\end{equation} 
Show that the vector $\vw_{\rm ML}$ solves the empirical risk minimization problem using logistic loss $L((\vx,y); \vw) = \ln\big(1 + \exp\big(- y (\vw^{T} \vx)\big))\big)$, i.e., $\vw_{\rm ML}$ is 
a solution to 
\vspace*{-2mm}
$$\min\limits_{\vw \in \mathbb{R}^{2}} (1/\samplesize) \sum_{\sampleidx=1}^{\samplesize} L((\vx^{(\sampleidx)},y^{(\sampleidx)}); \vw).$$

 
\noindent {\bf Answer.}

\newpage

\section{Logistic Regression - II}
Consider a binary classification problem where the goal is classify or label a webcam snapshot into ``winter'' ($y=-1$) or ``summer'' ($y=1$) based on the feature vector 
$\vx=(x_{\rm g},1)^{T} \in \mathbb{R}^{2}$ with the image greenness $x_{\rm g}$. A particular classification 
method is logistic regression, where we classify a datapoint as $\hat{y}=1$ if $h^{(\vw)}(\vx)=\sigma(\vw^T\vx) > 1/2$ and $\hat{y}=-1$ otherwise. Here, we used the sigmoid function 
$\sigma(z) = 1/(1+\exp(-z))$. 

Given some labeled snapshots $\dataset = \left\lbrace (x^{\sampleidx)}, y^{(\sampleidx)}) \right\rbrace_{\sampleidx=1}^{\samplesize}$, we choose the weight vector $\vw$ 
by empirical risk minimization using logistic loss $L((\vx,y); \vw) = \ln\big(1\!+\!\exp\big(- y (\vw^{T} \vx)\big)\big)$, i.e., 
\begin{equation}
\vw_{\rm opt} = \arg \min\limits_{\vw \in \mathbb{R}^{2}} \underbrace{(1/\samplesize) \sum_{\sampleidx=1}^{\samplesize} L((\vx^{(\sampleidx)},y^{(\sampleidx)}); \vw)}_{=f(\mathbf{w})}.
\end{equation} 
Since there is no simple closed-form expression for $\vw_{\rm opt}$, we have to use some optimization method for (approximately) finding $\vw_{\rm opt}$. One extremely useful such 
method is gradient descent which starts with some initial guess $\vw^{(0)}$ and iterates 
\begin{equation}
\vw^{(k+1)} = \vw^{(k)} - \alpha \nabla f(\mathbf{w}^{(k)}), 
\end{equation}
for $k=0,1,\ldots$. For a suitably chosen step-size $\alpha >0$ one can show that $\lim_{k \rightarrow \infty} \vw^{(k)} = \vw_{\rm opt}$. Can you find a simple closed-form expression 
for the gradient $\nabla f(\mathbf{w}^{(k)})$ in terms of the current iterate $\vw^{(k)}$ and the data points $\dataset = \left\lbrace (x^{(\sampleidx)}, y^{(\sampleidx)}) \right\rbrace_{\sampleidx=1}^{\samplesize}$.  

 
\noindent {\bf Answer.}

% ANSWER 2 HERE!
\centering
$\bigtriangledown f(\textbf{w}^{(k)})$ is a derivation of $f(\textbf{w}^{(k)})$\vspace{5mm}

$\bigtriangledown f(\textbf{w}^{(k)}) = \frac{df}{dw}\frac{1}{N}\sum\limits^N_{i=1}L((x^{(i)},y^{(i)}; w^{(k)}$\vspace{5mm}

$= \frac{df}{dw}\frac{1}{N}\sum\limits^N_{i=1}ln(1+exp(-y(i)(w^{(k)T}x^{(i)})))$\vspace{5mm}

A matrix expression is applied on X and Y.\vspace{5mm}

$\bigtriangledown f(\textbf{w}^{(k)}) =\frac{df}{dw}\frac{1}{N}ln(1+exp(- Y (w^T X))$\vspace{5mm}

$=\frac{1}{N} \frac{\frac{df}{dw}(1+exp(- Y (w^T X))}{1+exp(- Y (w^T X)}$\vspace{5mm}

$=\frac{1}{N} \frac{-YX \cdot exp(- Y (w^T X))}{1+exp(- Y (w^T X)}$\vspace{5mm}

$=\frac{-YX}{N(exp(Y (w^T X)+1)}$\vspace{5mm}

$=-\frac{YX}{N \cdot exp(Y (w^T X)+N}$\vspace{5mm}

% ANSWER 2 END!
\flushleft
\newpage

\section{Bayes' Classifier - I}
Consider a binary classification problem where the goal is classify or label a webcam snapshot into ``winter'' ($y=-1$) or ``summer'' ($y=1$) based on the feature vector 
$\vx=(x_{\rm g},x_{\rm r})^{T} \in \mathbb{R}^{2}$ with the image greenness $x_{\rm g}$ and redness $x_{\rm r}$. We might interpret 
the feature vector and label as (realizations) of random variables, whose statistics is specified by a joint distribution $p(\vx,y)$. This joint distribution factors as $p(\vx,y) = p(\vx| y) p(y)$ 
with the conditional distribution $p(\vx| y)$ of the feature vector given the true label $y$ and the prior distribution $p(y)$ of the label values. The prior probability $p(y=1)$ is the fraction of overall 
summer snapshots. Assume that we know the distributions $p(\vx| y)$ and $p(y)$ and we want to construct a classifier $h(\vx)$, which classifies a snapshot with feature vector $\vx$ as $\hat{y}=h(\vx) \in \{-1,1\}$. 
Which classifier map $h(\cdot): \vx \mapsto \hat{y}=h(\vx)$, mapping the feature vector $\vx$ to a predicted label $\hat{y}$, yields the smallest error probability (which is $p( y \!\neq\! h(\vx))$)? 
 
\noindent {\bf Answer.}

% ANSWER 3 HERE!

% ANSWER 3 END!

\newpage

\section{Bayes' Classifier - II}
Reconsider the binary classification problem of Problem 3, where the goal is classify or label a webcam snapshot into ``winter'' ($y=-1$) or ``summer'' ($y=1$) based on the feature vector 
$\vx=(x_{\rm g},x_{\rm r})^{T} \in \mathbb{R}^{2}$ with the image greenness $x_{\rm g}$ and redness $x_{\rm r}$. While in Problem 3 we assumed perfect knowledge 
of the joint distribution $p(\vx,y)$ of features $\vx$ and label $y$ (which are modelled as random variables), now we consider only knowledge of the prior probability $P(y=1)$, which we denote $P_{1}$. 
A useful ``guess'' for the distribution of the features $\vx$, given the label $y$, is via a Gaussian distribution. Thus, we assume 
\begin{equation}
p(\vx|y=1;\mathbf{m}_{s},\mathbf{C}_{s}) = \frac{1}{\sqrt{\det\{ 2 \pi \mathbf{C}_{s} \}}} \exp(-(1/2) (\vx\!-\!\vm_{s})^{T} \mathbf{C}_{s}^{-1} (\vx\!-\!\vm_{s})) \nonumber
\end{equation}
and, similarly, 
\begin{equation}
p(\vx|y=-1;\mathbf{m}_{w},\mathbf{C}_{w}) = \frac{1}{\sqrt{\det\{ 2 \pi \mathbf{C}_{w} \}}} \exp(-(1/2) (\vx\!-\!\vm_{w})^{T} \mathbf{C}_{w}^{-1} (\vx\!-\!\vm_{w})).  \nonumber
\end{equation} 
How would you choose (fit) the parameters $\mathbf{m}_{s},\mathbf{m}_{w} \in \mathbb{R}^{2}$ and $\mathbf{C}_{s},\mathbf{C}_{w} \in \mathbb{R}^{2 \times 2}$ 
for (to) a given labeled dataset $\dataset = \{ (\vx^{(\sampleidx)},y^{(\sampleidx)}) \}_{\sampleidx=1}^{\samplesize}$.  

\noindent {\bf Answer.}

% ANSWER 4 HERE!

Not much to read here, sorry. I hope you did well on your assignment! :)

% ANSWER 4 END!

\newpage
\section{Support Vector Classifier}
Consider data points with features $\vx^{(\sampleidx)} \in \mathbb{R}^{2}$ and labels $y^{(\sampleidx)} \in \{-1,1\}$. 
In the figures below, the data points with $y^{(\sampleidx)}=1$ are depicted as red crosses 
and the data points with  $y^{(\sampleidx)}=-1$ are depicted as blue filled circles. 
Which of the four figures depicts a decision boundary which could have been generated by a SVC. Justify your selection.
	
	\begin{figure}[ht!]
		\begin{center}
			\subfigure[]{%
				\includegraphics[width=0.2\textwidth]{SVM_1.PNG}
			}%
			\subfigure[]{%
				\includegraphics[width=0.2\textwidth]{SVM_2.PNG}
			}
			\subfigure[]{%
				\includegraphics[width=0.2\textwidth]{SVM_3.PNG}
			}%
			\subfigure[]{%
				\includegraphics[width=0.2\textwidth]{SVM_4.PNG}
			}
		\end{center}
	\end{figure}
\noindent {\bf Answer.}

% ANSWER 5 HERE!
\flushleft
Wikipedia summarises it pretty well (Linear SVM):

"If the training data are linearly separable, we can select two parallel hyperplanes that separate the two classes of data, so that the distance between them is as large as possible."\newline

Based on the above, picture (c) seems the most likely - (d) is non-linear and the margin between the data is not maximized in (a) or (b).
% ANSWER 5 END!

\end{document}
